\documentclass[11pt]{article}
\begin{document}
\title{An example of \texttt{r2tex}
\author{R2tex}}
\maketitle

\begin{figure}[h]
\begin{center}
\begin{small}
\input{fib.tex}
\end{small}
\end{center}
\caption{Fib on Linux stations}
\end{figure}

\begin{figure}[h]
\begin{center}
\begin{small}
\input{fib2.tex}
\end{small}
\end{center}
\caption{Fib on Linux stations (Ratio with respect to a Pentium 233)}
\end{figure}

\begin{figure}[h]
\begin{center}
\begin{small}
\input{fib4.tex}
\end{small}
\end{center}
\caption{Fib on Linux stations (Absolute + Ratio with respect to a Pentium 233)}
\end{figure}

\begin{figure}[h]
\begin{center}
\begin{small}
\input{fib3.tex}
\end{small}
\end{center}
\caption{Fib on Linux stations (percentage with respect to a Pentium 233)}
\end{figure}

\begin{figure}[h]
\begin{center}
\begin{small}
\input{fib5.tex}
\end{small}
\end{center}
\caption{Fib on Linux stations (absolute + percentage with respect to a Pentium 233)}
\end{figure}


\end{document}


